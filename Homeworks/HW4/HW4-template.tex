\documentclass{article}

\setlength{\headsep}{0.75 in}
\setlength{\parindent}{0 in}
\setlength{\parskip}{0.1 in}

%=====================================================
% Add PACKAGES Here (You typically would not need to):
%=====================================================

\usepackage{xcolor}
\usepackage[margin=1in]{geometry}
\usepackage{amsmath,amsthm}
\usepackage{fancyhdr}
\usepackage{enumitem}
\usepackage{graphicx}

%=====================================================
% Ignore This Part (But Do NOT Delete It:)
%=====================================================

\theoremstyle{definition}
\newtheorem{problem}{Problem}
\newtheorem*{fun}{Fun with Algorithms}
\newtheorem*{challenge}{Challenge Yourself}
\def\fline{\rule{0.75\linewidth}{0.5pt}}
\newcommand{\finishline}{\begin{center}\fline\end{center}}
\newtheorem*{solution*}{Solution}
\newenvironment{solution}{\begin{solution*}}{{\finishline} \end{solution*}}
\newcommand{\grade}[1]{\hfill{\textbf{($\mathbf{#1}$ points)}}}
\newcommand{\thisdate}{\today}
\newcommand{\thissemester}{\textbf{Rutgers: Spring 2021}}
\newcommand{\thiscourse}{CS 344: Design and Analysis of Computer Algorithms} 
\newcommand{\thishomework}{Number} 
\newcommand{\thisname}{Name} 
\newcommand{\thisextension}{Yes/No} 

\headheight 40pt              
\headsep 10pt
\renewcommand{\headrulewidth}{0pt}
\lhead{\small \textbf{Only for the personal use of students registered in CS 344, Spring 2021 at Rutgers University. Redistribution out of this class is strictly prohibited.}}
\pagestyle{fancy}

\newcommand{\thisheading}{
   \noindent
   \begin{center}
   \framebox{
      \vbox{\vspace{2mm}
    \hbox to 6.28in { \textbf{\thiscourse \hfill \thissemester} }
       \vspace{4mm}
       \hbox to 6.28in { {\Large \hfill Homework \#\thishomework \hfill} }
       \vspace{2mm}
         \hbox to 6.28in { { \hfill \thisdate  \hfill} }
       \vspace{2mm}
       \hbox to 6.28in { \emph{Name: \thisname \hfill Extension: \thisextension}}
      \vspace{2mm}}
      }
   \end{center}
   \bigskip
}

%=====================================================
% Some useful MACROS (you can define your own in the same exact way also)
%=====================================================


\newcommand{\ceil}[1]{{\left\lceil{#1}\right\rceil}}
\newcommand{\floor}[1]{{\left\lfloor{#1}\right\rfloor}}
\newcommand{\prob}[1]{\Pr\paren{#1}}
\newcommand{\expect}[1]{\Exp\bracket{#1}}
\newcommand{\var}[1]{\textnormal{Var}\bracket{#1}}
\newcommand{\set}[1]{\ensuremath{\left\{ #1 \right\}}}
\newcommand{\poly}{\mbox{\rm poly}}


%=====================================================
% Fill Out This Part With Your Own Information:
%=====================================================


\renewcommand{\thishomework}{3} %Homework number
\renewcommand{\thisname}{Ryan Coslove} % Enter your name here
\renewcommand{\thisextension}{Yes} % Pick only one of the two options accordingly

\begin{document}

\thisheading
\vspace{-0.75cm}


%=====================================================
% LaTeX Tip: You can erase this part from here.... 
%=====================================================		

\subsection*{Homework Policy}
\begin{itemize}
\item If you leave a question completely blank, you will receive 25\% of the grade for that question. This however does not apply to the extra credit questions.
\item You are allowed to discuss the homework problems with other students in the class. \textbf{But you must write your solutions independently.} 
You may also consult all the materials used in this course (video recordings, notes, textbook, etc.) while writing your solution, but no other resources are allowed.
\item Do not forget to write down your name and whether or not you are using one of your two extensions. Submit your homework on Canvas. 
\item Unless  specified otherwise, you may use any algorithm covered in class as a ``black box'' -- for example you can simply write ``use Prim's or Kruskal's algorithm to find an MST of the input graph in $O(m\log{m})$ time''. 
You are \textbf{strongly encouraged to use graph reductions} instead of designing an algorithm from scratch whenever possible (even when the question does not ask you to do so explicitly). 

\item Remember to always \textbf{prove the correctness} of your algorithms and \textbf{analyze their running time} (or any other efficiency measure asked in the question). 

\item The ``Challenge yourself'' and ``Fun with algorithms''  are both extra credit. These problems are significantly more challenging than the standard problems you see in this course (including lectures, homeworks, and exams). 
As a general rule, only attempt to solve these problems if you  enjoy them. 
\end{itemize}

\finishline

\begin{problem}
You are given the map of $n$ cities with $m$ bidirectional roads between different cities. You are asked to construct airports in some of the cities such that each city either has an airport itself or there is a way to go from this city to a city with an airport 
using the given roads---moreover, you can also construct a road between any two cities if there is no road between them already. Finally, you are told that the cost of constructing an airport is $a$ and the cost of connecting any two cities 
by a new road is $r$. 

Design and analyze an $O(n+m)$ time algorithm that given the number $n$ of the cities, the $m$ roads between them, and the costs $a$ and $r$, outputs the locations of airports and roads to be constructed 
to satisfy the conditions above, while having the minimum possible cost. \grade{25}

\paragraph{Examples:}

\begin{enumerate}
    \item \emph{Input:} $n=4$ cities with $m=2$ roads $(1,2), (2,3)$, cost of constructing an airport $a = 7$, and constructing a road is $r=5$. 
    
   \emph{Output:} The minimum cost needed is 12 -- we construct an airport in city $4$ and connect it this city via a road to any of the cities $1$, $2$, or $3$ (chosen arbitrarily). 
 
    \item \emph{Input:} $n=4$ cities with $m=2$ roads $(1,2), (2,3)$, cost of constructing an airport $a = 7$, and constructing a road is $r=8$. 
    
   \emph{Output:} The minimum cost needed is 14 -- we construct an airport in city $4$ and another one in any one of the cities $1$, $2$, or $3$ (chosen arbitrarily). 

\end{enumerate}
\end{problem}

\begin{solution}
\textbf{Algorithm}
\item First we will use DFS to find the number of connected cities, and we know that DFS runs an average of O(n+m) time.
\item. Second, we can arbitrarily place an airport at any city, which will cost $a$.
\item If we make the number of connected cities as $c$, we can use the following as a way to find the minimum: $a + (c - 1) * min\{a, r\}$

\item \textbf{Proof of correctness}
\item If we use the first example as a base case of  $n=4$ cities with $m=2$ roads $(1,2), (2,3)$, cost of constructing an airport $a = 7$, and constructing a road is $r=5$ , let's check if the formula works.
\item Given the roads, 2 is connected to 1 and 3 so $c = 2$. Using our equation we have $7 + (2 - 1) * min\{7, 5\}$. This would equal $7 + 1*(5) = 12$, which is expected.
\item If we use the second example given, the equation is $7 + (2 - 1) * min\{7, 8\}$. This is would equal  $7 + 1*(7) = 14$, which is expected.

\item \textbf{Runtime}
\item Using DFS to find the number of connected cities will use a run time of $O(n + m)$. Implementing the equation that will tell us the minimum possible cost will take $O(1)$ time.
\item So, total runtime is $O(n+m)$.
 
\end{solution}



\begin{problem}
	We say that an undirected graph $G=(V,E)$ is \textbf{$2$-edge-connected} if we need to remove \emph{at least two} edges from $G$ to make it disconnected. Prove that a 
	graph $G=(V,E)$ is $2$-edge-connected if and only if for every cut $(S,V-S)$ in $G$, there are \emph{at least two cut edges}, i.e., $|{\delta(S)}| \geq 2$. \grade{25}
\end{problem}


\medskip

\begin{solution}
\item Since the given graph is undirected, the problem can be solved only by counting the number of edges connected to the nodes. If for any of the nodes, the number of edges connected to it is 1 it means on removing this edge the node becomes disconnected and it can’t be reached from any other node therefore the graph is not 2-edge connected. Here are the steps: 
\begin{enumerate}
\item Create an array $noOfEdges[]$ of size $V$ which will store the number of edges connected to a node.
\item For every edge $(u, v)$ increment the number of edges for node u and v.
\item Now iterate over the array $noOfEdges[]$ and check if any of the edge has only 1 edge connected to it. If yes then the graph is not 2-edge connected.
\item Otherwise, the graph is 2-edge connected.
\end{enumerate}
\item\textbf{Proof}
\item If our graph does not have any edges with only 1 edge connection, we can confirm that $|\delta(S)| \geq 2$ 
\item Any connected graph with at least two vertices can be disconnected by removing edges: by removing all edges incident with a single vertex the graph is disconnected. Thus, $\lambda(G) \leq \sigma(G)$, where $\sigma(G)$ is the minimum degree of any vertex in G. Note that $\sigma(G) \leq n-1$, so $\lambda(G) \leq n-1$.
\item Removing a vertex also removes all of the edges incident with it, which suggests that $k(G) \leq \lambda(G)$. This turns out to be true. We write $G-v$ to mean $G$ with vertex $v$ removed, and $G-\{v_1,v_2,…,v_k\}$ (k is cutsize) to mean $G$ with all of $\{v_1,v_2,…,v_k\}$ removed, and similarly for edges.
\end{solution}


\medskip

\begin{problem}
	The Muddy City consists of $n$ houses but no proper streets; instead, the different houses are connected to each other via $m$ bidirectional muddy roads. The newly elected mayor of the city aims to pave some of these roads to ease
	the travel inside the city but also does not want to spend too much money on this project, as paving each road $e$ between houses $u$ and $v$ has a certain cost $c_e$ (different across the muddy roads). The mayor thus specifies two conditions: 
	\begin{itemize}
		\item Enough streets must be paved so that everyone can travel from any house to another one using only the paved roads (you may assume that this is always possible); 
		\item The paving should cost as little as possible. 
	\end{itemize}
	You are chosen to help the mayor in this endeavor.  
	
	\begin{enumerate}
		\item [(a)] Design and analyze an $O(m\log{m})$ time algorithm for this problem. \grade{10} 
		
		\medskip

\begin{solution}
\item We will use Kruskal's algorithm to create a MST constructed where it starts with the lowest cost edge, then continually adding the next lowest cost edge/road until all vertices have been reached. 
\item This is the algorithm:
\begin{enumerate}
\item Sort the edges of $G$ in increasing (non-decreasing) order of their weights.
\item Let $F = \emptyset$.
\item For $i = 1$ to $m$ (in the sorted ordering of edges): If adding $c_e$ to $F$ does not create a cycle, let $F \leftarrow F \cup \{c_e\}$.
\item Return F
\end{enumerate}

\item\textbf{Proof}
\item The proof of correctness is done in lecture for Kruskal's algorithm. Consider the forest $F$ maintained by the algorithm.  We prove that if F is MST-good at some iteration i and in that iteration we inserted an edge $c_e$ to F, then $c_e$ was safe with respect to F.   This  then  implies  that  we  only  added  safe  edges to F. Moreover,  the  output  of  this  algorithm  isalways a tree since whenever we see an edge that does not create a cycle we add it to F (and since G was connected, we will find a tree eventually this way).  That means that we added $n-1$ safe edges.  This would imply the correctness of the algorithm as we now can say that this algorithm is an implementation of the meta-algorithm. Since the edges are sorted in non-decreasing order of their weights, we have that $c_e$ has the minimum weight among cut edges of $(S,V-S)$. We are now done since we can apply Theorem 1 and argue that $c_e$ is a safe edge.  This finalizes the proof.

\item \textbf{Runtime}
\item The first line of the algorithm takes $O(m\log m)$ time to sort all the edges (using, say, merge sort).  We then have m iteration, each iteration involve deciding whether adding the edge $c_e$ to the graph creates a cycle or not.  The easiest way to implement this step is to run DFS/BFS - that will take $O(n+m)$ time per each iteration, making the total runtime of the algorithm $O(m*(n+m)) = O(mn+m^2) = O(m^2)$ since we know that $m \geq n-1$ as G was connected (any connected graph needs at least $n-1$ edges). Thus, overall, the runtime of the algorithm is $O(m\log m)$.
\end{solution}

		\item [(b)] The mayor of a neighboring city is feeling particularly generous and has made the following offer to Muddy City: they have identified a list of $O(\log{m})$ different muddy roads in the city and are willing to entirely pay
		 the cost of paving \emph{exactly one} of them (in exchange for calling the new street after the neighboring city). 
		 
		 Design and analyze an $O(m\log{m})$ time algorithm that identifies the paving of which of these roads, if any, the mayor should delegate to the neighboring city 
		 to further minimize the total cost---note that if you  decide to pave one of the roads payed by the neighboring city, you only need to pay a cost of $1$ (for making a plaque of the name of the street). \grade{15}
		 
		 \smallskip
		 \emph{Hint:} Design an algorithm that given a MST $T$ of a graph $G$, and a single edge $e$, in only $O(m)$ time finds an MST $T'$ for the graph $G'$ obtained by changing the weight of the edge $e$ to $1$. 
		 
		 \medskip

\begin{solution}
\item We will use a similar approach to part (a). We will use Kruskal's algorithm again, and use a sorting algorithm like merge sort to order them in increasing (non-decreasing order).  
\begin{enumerate}
\item Sort the edges of $G$ in increasing (non-decreasing) order of their weights.
\item Let $F = \emptyset$.
\item For $i = 1$ to $m$ (in the sorted ordering of edges): If adding $c_e$ to $F$ does not create a cycle, let $F \leftarrow F \cup \{c_e\}$.
\item Return F
\item The next step to the algorithm is to find the most expensive road $c_e$. We will iterate through the array of $c_e$'s and find the max value (the last safe edge which will be $e$).  
\item We create an MST T' for the graph G' by changing the weight/cost of this edge $e$ to 1. 
\end{enumerate}
\item Now, we have our new MST T' for graph G' which has taken the most expensive road $e$ from MST T of graph G and replaced it with a value of 1. 

\item \textbf{Proof of correctness}
\item The proof for the Kruskal's algorithm and its correctness is done in part (a). In addition to that proof, we know this is correct because we have iterated through the sorted list of $c_e$ values, found the most expensive (maximum value) road and have given it to the neighboring city, thus reducing the cost of G and producing G'. For example if our G contained $c_e$'s of \{1, 2, 3, 4, 5\} whose total cost is 15, we search for the highest value (5 in this case) and change it to 1 and produce G'. G' will now contain $c_e$'s of \{1, 1, 2, 3, 4\}  whose total cost is 11. With the knowledge that G' always produces a lower total cost than G, we prove that our algorithm works and that we will have the minimum possible cost.

\item\textbf{Runtime}
\item The first line of the algorithm takes $O(m\log m)$ time to sort all the edges (using, say, merge sort).  We then have m iteration, each iteration involve deciding whether adding the edge $c_e$ to the graph creates a cycle or not.  The easiest way to implement this step is to run DFS/BFS - that will take $O(n+m)$ time per each iteration, making the total runtime of the algorithm $O(m*(n+m)) = O(mn+m^2) = O(m^2)$ since we know that $m \geq n-1$ as G was connected (any connected graph needs at least $n-1$ edges). It also only takes $O(m)$ time to find an MST' for graph G'. Thus, overall, the runtime of the algorithm is $O(m\log m)$.

\end{solution}
\end{enumerate}	
\end{problem}

\medskip


\begin{problem}
	You are given a weighted undirected graph $G=(V,E)$ with integer weights $w_e \in \set{1,2,\ldots,W}$ on each edge $e$, where $W=O(1)$. 
	Given two vertices $s,t \in V$, the goal is to find the minimum weight path (or shortest path) from $s$ to $t$. Recall that Dijkstra's algorithm solves this problem in $O(n+m\log{m})$ time even if we do not have the condition that $W=O(1)$. 
	However, we now want to use this extra condition to design an even faster algorithm. 
	
	
	Design and analyze an algorithm to find the minimum weight (shortest) $s$-$t$ path in $O(n+m)$ time.
\end{problem}

\medskip

\begin{solution}
\item We will utilize BFS for this problem. This is the algorithm:
\begin{enumerate}
\item Initialize an array $mark[1 : t]$ with ‘FALSE’ and $s$ be the designated source vertex.
\item Create a queue data structure, set $mark[s] = 'TRUE'$ and let $S$ (initially)  be  the  set  of  edges  incident  on $s$ and  assign  a  value $value(e) =d[s] + w_e$ to each of these edges.
\item While S is not empty:
\begin{itemize}
\item  Let $e= (u, v)$ be the minimum value edge in $S$ and remove $e$ from $S$.
\item If $mark[v] = ‘TRUE’$; continue to the next iteration of the while-loop.
\item Otherwise, let $mark[v] = ‘TRUE’$ and $d[v] =value(e)$, and insert all edges $e'$ incident on $v$ to $S$ with $value(e') = d[v] + w'_e$.
\end{itemize}
\item Return $d$
\end{enumerate}

\item \textbf{Proof}
\item The proof of BFS is done in lecture. The alteration to this from our normal BFS is we are queueing the integer weights, iterating through the minimum weights in the queue. 

\item \textbf{Runtime}
\item Because we are calling this algorithm on disconnected sub graphs, doing each one once, and knowing the general time complexity of running a BFS, we can  assure that to find the minimum weight (shortest) $s-t$ path is in $O(n+m)$ time.

\end{solution}

\smallskip



\smallskip

\begin{challenge}
	A \textbf{bottleneck spanning tree (BST)} of a undirected connected graph $G=(V,E)$ with positive weights $w_e$ over each edge $e$, is a spanning tree $T$ of $G$ such that the weight of maximum-weight edge in $T$ is minimized 
	across all spanning trees of $G$. In other words, if we define the cost of $T$ as $\max_{e \in T} w_e$, a BST has minimum cost across all spanning trees of $G$. Design and analyze an $O(n+m)$ time algorithm for finding a BST of a given graph. 
	  \grade{+10}
\end{challenge}

\smallskip

\begin{fun}
	Let us go back to the bottleneck spanning tree (BST) problem defined above. 
	\begin{enumerate}
		\item[(a)] Prove that any MST of any graph $G$ is also a BST. Use this to obtain an $O(m\log{m})$ time algorithm for the BST problem (notice that this is  slower than the algorithm from the previous question). 
		
		\grade{+8}
		\item[(b)] Give an example of a BST of some graph $G$ which is \emph{not} an MST in $G$. \grade{+2}
	\end{enumerate}
 \end{fun}


\end{document}





